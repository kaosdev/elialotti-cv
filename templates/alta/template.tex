%% Use the "normalphoto" option if you want a normal photo instead of cropped to a circle
% \documentclass[10pt,a4paper,normalphoto]{altacv}

\documentclass[10pt,a4paper,ragged2e,withhyper]{altacv}
%% AltaCV uses the fontawesome5 and packages.
%% See http://texdoc.net/pkg/fontawesome5 for full list of symbols.

% Change the page layout if you need to
\geometry{left=1.25cm,right=1.25cm,top=1.5cm,bottom=1.5cm,columnsep=1.2cm}

% The paracol package lets you typeset columns of text in parallel
\usepackage{paracol}

% Change the font if you want to, depending on whether
% you're using pdflatex or xelatex/lualatex
% WHEN COMPILING WITH XELATEX PLEASE USE
% xelatex -shell-escape -output-driver="xdvipdfmx -z 0" sample.tex
\ifxetexorluatex
  % If using xelatex or lualatex:
  \setmainfont{Roboto Slab}
  \setsansfont{Lato}
  \renewcommand{\familydefault}{\sfdefault}
\else
  % If using pdflatex:
  \usepackage[rm]{roboto}
  \usepackage[defaultsans]{lato}
  % \usepackage{sourcesanspro}
  \renewcommand{\familydefault}{\sfdefault}
\fi

% Change the colours if you want to
\definecolor{SlateGrey}{HTML}{2E2E2E}
\definecolor{LightGrey}{HTML}{666666}
\definecolor{DarkPastelRed}{HTML}{450808}
\definecolor{PastelRed}{HTML}{8F0D0D}
\definecolor{GoldenEarth}{HTML}{E7D192}
\colorlet{name}{black}
\colorlet{tagline}{PastelRed}
\colorlet{heading}{DarkPastelRed}
\colorlet{headingrule}{GoldenEarth}
\colorlet{subheading}{PastelRed}
\colorlet{accent}{PastelRed}
\colorlet{emphasis}{SlateGrey}
\colorlet{body}{LightGrey}

% Change some fonts, if necessary
\renewcommand{\namefont}{\Huge\rmfamily\bfseries}
\renewcommand{\personalinfofont}{\footnotesize}
\renewcommand{\cvsectionfont}{\LARGE\rmfamily\bfseries}
\renewcommand{\cvsubsectionfont}{\large\bfseries}


% Change the bullets for itemize and rating marker
% for \cvskill if you want to
\renewcommand{\cvItemMarker}{{\small\textbullet}}
\renewcommand{\cvRatingMarker}{\faCircle}
% ...and the markers for the date/location for \cvevent
% \renewcommand{\cvDateMarker}{\faCalendar*[regular]}
% \renewcommand{\cvLocationMarker}{\faMapMarker*}


% If your CV/résumé is in a language other than English,
% then you probably want to change these so that when you
% copy-paste from the PDF or run pdftotext, the location
% and date marker icons for \cvevent will paste as correct
% translations. For example Spanish:
% \renewcommand{\locationname}{Ubicación}
% \renewcommand{\datename}{Fecha}

\begin{document}
\name{<var>general.name</var>}
\tagline{<var>general.jobtitle</var>}
%% You can add multiple photos on the left or right
% \photoR{2.8cm}{Globe_High}
% \photoL{2.5cm}{Yacht_High,Suitcase_High}

\personalinfo{%
  % Not all of these are required!
  \email{<var>general.email</var>}
  % \phone{000-00-0000}
  % \mailaddress{Åddrésş, Street, 00000 Cóuntry}
  \location{<var>general.location</var>}
  \homepage{<var>general.website</var>}
  % \twitter{@twitterhandle}
  \linkedin{<var>general.linkedin</var>}
  \github{<var>general.github</var>}
  % \orcid{0000-0000-0000-0000}
  %% You can add your own arbitrary detail with
  %% \printinfo{symbol}{detail}[optional hyperlink prefix]
  % \printinfo{\faPaw}{Hey ho!}[https://example.com/]

  %% Or you can declare your own field with
  %% \NewInfoFiled{fieldname}{symbol}[optional hyperlink prefix] and use it:
  % \NewInfoField{gitlab}{\faGitlab}[https://gitlab.com/]
  % \gitlab{your_id}
  %%
  %% For services and platforms like Mastodon where there isn't a
  %% straightforward relation between the user ID/nickname and the hyperlink,
  %% you can use \printinfo directly e.g.
  % \printinfo{\faMastodon}{@username@instace}[https://instance.url/@username]
  %% But if you absolutely want to create new dedicated info fields for
  %% such platforms, then use \NewInfoField* with a star:
  % \NewInfoField*{mastodon}{\faMastodon}
  %% then you can use \mastodon, with TWO arguments where the 2nd argument is
  %% the full hyperlink.
  % \mastodon{@username@instance}{https://instance.url/@username}
}

\makecvheader
%% Depending on your tastes, you may want to make fonts of itemize environments slightly smaller
% \AtBeginEnvironment{itemize}{\small}

%% Set the left/right column width ratio to 6:4.
\columnratio{0.6}

% Start a 2-column paracol. Both the left and right columns will automatically
% break across pages if things get too long.
\begin{paracol}{2}
  \cvsection{Esperienza}

  <block>for job in experience</block>
  \cvevent{<var>job.title</var>}{<var>job.company</var>}{<var>job.start</var> -- <var>job.end</var>}{<var>job.location</var>}
  \begin{itemize}
    <block>for desc in job.descriptions</block>
    \item <var>desc</var>
    <block>endfor</block>
  \end{itemize}

  <block>if not loop.last</block>

  \divider

  <block>endif</block>

  <block>endfor</block>

  \cvsection{Progetti Personali}

  <block>for proj in projects</block>

  \cvevent{<var>proj.name</var>}{}{}{}
  <var>proj.desc</var>

  <block>if not loop.last</block>

  \divider

  <block>endif</block>

  <block>endfor</block>

  \medskip

  %% Switch to the right column. This will now automatically move to the second
  %% page if the content is too long.
  \switchcolumn

  \cvsection{Skills}

  <block>for _, section_skills in skills.items()</block>
    <block>for skill in section_skills</block>\cvtag{<var>skill</var>}<var>"\\\\" if (loop.index % 3) == 0 else ""</var><block>endfor</block>
  <block>if not loop.last</block>
  \divider\smallskip
  <block>endif</block>
  <block>endfor</block>

  \cvsection{Lingue}

  <block>for lang, level in languages.items()</block>

  \cvskill{<var>lang</var>}{<var>level</var>}
  \divider

  <block>endfor</block>

  %% Yeah I didn't spend too much time making all the
  %% spacing consistent... sorry. Use \smallskip, \medskip,
  %% \bigskip, \vspace etc to make adjustments.
  \medskip

  \cvsection{Certificazioni}

  <block>for cert in certifications</block>

  \cvevent{<var>cert.title</var>}{<var>cert.academy</var>}{<var>cert.date</var>}{}

  <block>if cert.link</block>

    \href{<var>cert.link</var>}{<var>cert.link</var>}

  <block>endif</block>
  
  <block>if not loop.last</block>

  \divider

  <block>endif</block>

  <block>endfor</block>

  \medskip

  \cvsection{Istruzione}

  <block>for edu in education</block>

  \cvevent{<var>edu.title</var>}{<var>edu.academy</var>}{<var>edu.date</var>}{}

  <block>if not loop.last</block>

  \divider

  <block>endif</block>

  <block>endfor</block>

\end{paracol}


\end{document}
